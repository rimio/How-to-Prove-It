\documentclass[12pt]{amsart}

%Below are some necessary packages for your course.
\usepackage{amsfonts,latexsym,amsthm,amssymb,amsmath,amscd,euscript}
\usepackage{framed}
\usepackage{fullpage}
\usepackage{hyperref}
    \hypersetup{colorlinks=true,citecolor=blue,urlcolor =black,linkbordercolor={1 0 0}}
\usepackage{mathtools}

\newenvironment{statement}[1]{\smallskip\noindent\color[rgb]{.6627, .3529, .6314} {\bf #1.}}{}
\allowdisplaybreaks[1]

%Below are the theorem, definition, example, lemma, etc. body types.

\newtheorem{theorem}{Theorem}
\newtheorem*{proposition}{Proposition}
\newtheorem{lemma}[theorem]{Lemma}
\newtheorem{corollary}[theorem]{Corollary}
\newtheorem{conjecture}[theorem]{Conjecture}
\newtheorem{postulate}[theorem]{Postulate}
\theoremstyle{definition}
\newtheorem{defn}[theorem]{Definition}
\newtheorem{example}[theorem]{Example}

\theoremstyle{remark}
\newtheorem*{remark}{Remark}
\newtheorem*{notation}{Notation}
\newtheorem*{note}{Note}

% You can define new commands to make your life easier.
\newcommand{\BR}{\mathbb R}
\newcommand{\BC}{\mathbb C}
\newcommand{\BF}{\mathbb F}
\newcommand{\BQ}{\mathbb Q}
\newcommand{\BZ}{\mathbb Z}
\newcommand{\BN}{\mathbb N}

% We can even define a new command for \newcommand!
\newcommand{\nc}{\newcommand}

% If you want a new function, use operatorname to define that function (don't use \text)
\nc{\on}{\operatorname}
\nc{\Spec}{\on{Spec}}

\title{\emph{How to Prove It}: Introduction} % IMPORTANT: Change the problemset number as needed.
\date{\today}

\begin{document}

\maketitle

\vspace*{-0.25in}
\centerline{Kyle Stratton}

\begin{framed}
These are the exercises for the Introduction from the third edition of \emph{How to Prove It} by Daniel J. Velleman.
They are numbered I.(Exercise Number).
\end{framed}

\begin{statement}{I.1}
\begin{enumerate}
	\item Factor $2^{15} - 1 = 32,767$ into a product of two smaller positive integers.
	\item Find an integer $x$ such that $1 < x < 2^{32,767} - 1$ and $2^{32,767} - 1$ is divisible by $x$.
\end{enumerate}
\end{statement}

\begin{proof}
For this exercise, we follow the strategy for the proof of Conjecture I.2.
If $n$ is not prime, then we can factorize it as $n = ab$ for two positive integers $a$ and $b$ such that $a < n$, and $b < n$.
Letting $x = 2^b - 1$ and $y = 1 + 2^b + 2^{2b} + \dots + 2^{(a - 1)b} = \sum_{k = 0}^{a - 1} 2^{kb}$, we have $xy = 2^n - 1$.
\begin{enumerate}
	\item To factor $2^{15} = 32,767$, we note that $15 = 3 \times 5$.
	Letting $a = 3$ and $b = 5$, we set $x = 2^5 - 1 = 31$ and $y = 1 + 2^5 + 2^{2 \times 5} = 1057$.
	It is straightforward to check that $31 \times 1057 = 32,767$.
	
	\item From part (1), we saw that we could factor $36,767 = 31 \times 1057$.
	Therefore, we can repeat the same strategy to find factors of $2^{32,767} - 1$.
	This time we let $a = 1057$ and $b = 31$, to give us $x = 2^{31} - 1$ and $y = \sum_{k = 0}^{1056} 2^{31k}$.
\end{enumerate}
\end{proof}


\begin{statement}{I.2}
Make some conjectures about the values of $n$ for which $3^n - 1$ is prime or the values of $n$ for which $3^n - 2^n$ is prime.
\end{statement}

\begin{proof}
We create a table for $n = 2, \dots, 10$.
\begin{center}
	\begin{tabular}{c | l | c | l | c | l}
		$n$ & Is $n$ prime? & $3^n - 1$ & Is $3^n - 1$ prime?  & $3^n - 2^n$ & Is $3^n - 2^n$ prime?\\
		\hline
		2 & yes & 8 & no: $8 = 2 \times 4$ & 5 & yes \\
		3 & yes & 26 & no: $26 = 2 \times 13$ & 19 & yes \\
		4 & no: $4 = 2 \times 2$ & 80 & no: $80 = 2 \times 40$ & 65 & no: $65 = 5 \times 13$ \\
		5 & yes & 242 & no: $242 = 2 \times 121$ & 211 & yes \\
		6 & no: $6 = 2 \times 3$ & 728 & no: $728 = 2 \times 364$ & 665 & no: $665 = 5 \times 133$ \\
		7 & yes & 2186 & no: $2186 = 2 \times 1093$ & 2059 & no: $2059 = 29 \times 71$ \\
		8 & no: $8 = 2 \times 2$ & 6560 & no: $6560 = 2 \times 3280$ & 6305 & no: $6305 = 5 \times 1261$ \\
		9 & no: $9 = 3 \times 3$ & 19682 & no: $19682 = 2 \times 9841$ & 19171 & no: $19171 = 19 \times 1009$ \\
		10 & no: $10 = 2 \times 5$ & 59048 & no: $59048 = 2 \times 29524$ & 58025 & no: $58025 = 5 \times 11605$
	\end{tabular}
\end{center}
As we can see, it appears that we can conjecture first that $3^n - 1$ is never prime for $n > 2$ (when $n = 1$, $3^1 - 1 = 2$, which is prime).
We can also conjecture that when $n$ is not prime, $3^n - 2^n$ is not prime as well.
When $n$ is prime, $3^n - 2^n$ was prime in some instances, but not always.
\end{proof}


\begin{statement}{I.3}
The proof of the Theorem I.3 (there are infinitely many prime numbers) gives a method for finding a prime number different from any in a given list of prime numbers.
\begin{enumerate}
	\item Use this method to find a prime different from 2, 3, 5, and 7.
	\item Use this method to find a prime different from 2, 5, and 11.
\end{enumerate}
\end{statement}

\begin{proof}
Recall that in the proof of Theorem I.3 we supposed that we had a list $p_1, p_2, \dots, p_n$ of prime numbers.
We then let $m = p_1 p_2 \cdots p_n + 1$.
Either $m$ itself is prime, or $m$ is a product of primes which are not on the list, as $m \cong 1 \mod p_k$ for $k = 1, \dots, n$.
\begin{enumerate}
	\item Let $m = (2 \times 3 \times 5 \times 7)  + 1 = 211$.
	We can check by hand directly that 211 is prime.
	
	\item Let $m = (2 \times 5 \times 11) + 1 = 111$.
	We can directly check that $111 = 3 \times 37$, and that both 3 and 37 are prime.
\end{enumerate}
\end{proof}


\begin{statement}{I.4}
Find five consecutive integers that are not prime.
\end{statement}

\begin{proof}
We use the proof of Theorem I.4, which states that for every positive integer $n$, there is a sequence of $n$ consecutive positive integers containing no primes.
In the proof we constructed such a sequence, by letting $x = (n + 1)! + 2$ and then showing that none of the numbers $x, x + 1, x + 2, \dots, x + (n - 1)$ is prime.
For this exercise we use $n = 5$ to obtain $x = (5 + 1)! + 2 = 722$.
Then $722 = 2 \times 361$, $723 = 3 \times 241$, $724 = 4 \times 181$, $725 = 5 \times 145$, and $726 = 6 \times 121$
are five consecutive integers that are not prime.
\end{proof}


\begin{statement}{I.5}
Use the table in Figure I.1 and the discussion on p. 5 to find two more perfect numbers.
\end{statement}

\begin{proof}
Recall that the table in Figure I.1 was as follows.
\begin{center}
	\begin{tabular}{c | l | c | l}
		$n$ & Is $n$ prime? & $2^n - 1$ & Is $2^n - 1$ prime? \\
		\hline
		2 & yes & 3 & yes \\
		3 & yes & 7 & yes \\
		4 & no: $4 = 2 \times 2$ & 15 & no: $15 = 3 \times 5$ \\
		5 & yes & 31 & yes \\
		6 & no: $6 = 2 \times 3$ & 63 & no: $63 = 7 \times 9$ \\
		7 & yes & 127 & yes \\
		8 & no: $8 = 2 \times 2$ & 255 & no: $255 = 15 \times 17$ \\
		9 & no: $9 = 3 \times 3$ & 511 & no: $511 = 7 \times 73$ \\
		10 & no: $10 = 2 \times 5$ & 1023 & no: $1023 = 31 \times 33$
	\end{tabular}
\end{center}
The discussion on p. 5 states that if $2^n - 1$ is prime, then $2^{n - 1} (2^n - 1)$ is a perfect number.
Thus, two additional perfect numbers are $2^4 (2^5 - 1) = 496$ and $2^6 (2^7 - 1) = 8128$.
Indeed, the factors of 496 that are strictly less than 496 are 1, 2, 4, 8, 16, 31, 62, 124, and 248.
Adding them up gives a sum of 496.
Similarly, the factors of 8128 that are less than 8128 are 1, 2, 4, 8, 16, 32, 64, 127, 254, 508, 1016, 2032, and 4064.
Adding those factors together gives a sum of 8128.
\end{proof}


\begin{statement}{I.6}
The sequence 3, 5, 7 is a list of three prime numbers such that each pair of adjacent numbers in the list differ by two.
Are there any more such ``triplet primes''?
\end{statement}

\begin{proof}
We claim that $\{2, 3, 5\}$ and $\{3, 5, 7\}$ are the only trios of triplet primes.
To see why this is the case, consider an arbitrary trio of consecutive positive odd integers $x, x + 2, x + 4$ such that $x > 3$.
We then look at the remainders of $x, x + 2, x + 4$ modulo 3.
First, if $x \cong 0 \mod 3$, then $x$ is divisible by 3 and therefore not prime.
Next, if $x \cong 1 \mod 3$, then $x + 2 \cong 0 \mod 3$, so $x + 2$ is divisible by 3 and not prime.
Last, if $x \cong 2 \mod 3$, then $x + 4 \cong 0 \mod 3$, so $x + 4$ is divisible by 3 and not prime.
In all three cases, at least one of the numbers $x, x + 2, x + 4$ is divisible by 3.
In other words, $x, x + 2, \text{ and } x + 4$ cannot all be prime when $x > 3$.
\end{proof}


\begin{statement}{I.7}
A pair of distinct positive integers $(m, n)$ is called \emph{amicable} if the sum of all positive integers smaller than $n$ that divide $n$ is $m$, and the sum of all positive integers smaller than $m$ that divide $m$ is $n$.
Show that $(220, 284)$ is amicable.
\end{statement}

\begin{proof}
The factors of 220 are 1, 2, 4, 5, 10, 11, 20, 22, 44, 55, 110, and 220.
Adding the ones which are strictly smaller than 220, we get a sum of 284.
Similarly, the factors of 284 are 1, 2, 4, 71, 142, and 284.
Adding those which are strictly smaller than 284, we get a sum of 220.
Hence, the pair $(220, 284)$ is amicable.
\end{proof}
\end{document}